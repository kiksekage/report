% !TeX root = ../report.tex

\section{Conclusion}
In this report, two different models were presented that both solve subtasks in SemEval-2018 Task 1. Firstly, a feature based model was presented. This model acted as an opener for the task and possible solutions to it. It presented a decent baseline for the scores and the general solving of the task. The results were subpar with regards to last years contending models. The custom features that were implemented proved to be useful and this seemed to be the greatest takeaway from the approach.\\
The second model presented acted as the next step in solving the task. The implementation of the model was shown to be lackluster when it came to predicting the emotion intensities, but possible solutions to this complication were discussed in section \ref{sec:comparison}. This however was a part of the multi-task approach to the problem, in which a single model was capable of solving both tasks. The results from this approach showed an improvement in the classification task, but not resoundingly so in regression.\\
Testing of both of the models proved to be insightful, both with regards to the strengths and weaknesses of the two models, but also with regards to the shared task as a whole. Overall, the general focus was on the deep learning model, which also had the most potential, from a quality of score point-of-view as well as an insight into the different ways of tackling the presented shared task.\\
\\
The code for the feature based model can be found on\\
\href{https://github.com/kiksekage/casperogthomas}{https://github.com/kiksekage/casperogthomas}.\\
The code for the deep learning model can be found on \href{https://github.com/kiksekage/KU-MTL}{https://github.com/kiksekage/KU-MTL}.