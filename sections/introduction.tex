% !TeX root = ../report.tex
\section{Introduction}
\subsection{Classification}
In the field of Natural Language Processing, emotion detection in text is the process of classifying documents in the form of words or sentences to a certain discrete set of predetermined emotions. \\
In this project we are working with 4 different emotions, namely anger, fear, joy and sadness. Our goal is to correctly classify an unseen tweet to one of these 4 emotions. We do this by training a model on a set of pre-sampled annotated tweets from the sem-eval task. We try different models and different approaches to both the classification and regression task. Amongst others we are using linear classifiers and recurrent neural networks.
\subsection{Regression}
The regression part is similar in the sense that we are still using NLP - only now we are detecting emotion intensity instead of detection. The tweets we are training on are annotated with continuous regression scores between 0 and 1.