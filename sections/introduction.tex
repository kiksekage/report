% !TeX root = ../report.tex
\section{Introduction}
We communicate via natural language, but it is not only the message that is communicated, it is also the sentiment and intensity of that sentiment. Intensity of sentiment in sentences and words are borderline subjective and therefore difficult to classify in general. Therefore it is interesting to see if it is possible to build a model or program that can do this automatically. \\
Natural Language Processing (NLP) is the science regarding natural, "human" language, and in this field detecting emotion or sentiment in text can have a lot of real world uses, such as analysing stance towards political statements or online consumer reviews of products, etc. \\
Twitter is a great service for detecting these sentiments, since it is user-written and often emotionally fuelled(spelling?).
\subsection{Task} \label{sec:task}
The task to be solved is the SemEval-2018 Task 1: Affect in Tweets, and can be found on \hyperref[https://competitions.codalab.org/competitions/17333]{https://competitions.codalab.org/competitions/17333}. The task is split into five subtasks:\\
\begin{enumerate}
\item EI-reg (Given a tweet and an emotion E, determine the  intensity of E that best represents the mental state of the tweeter—a real-valued score between 0 (least E) and 1 (most E).)
\item EI-oc (Given a tweet and an emotion E, classify the tweet into one of four ordinal classes of intensity of E that best represents the mental state of the tweeter.)
\item V-reg (Given a tweet, determine the intensity of sentiment or valence (V) that best represents the mental state of the tweeter—a real-valued score between 0 (most negative) and 1 (most positive).)
\item V-oc (Given a tweet, classify it into one of seven ordinal classes, corresponding to various levels of positive and negative sentiment intensity, that best represents the mental state of the tweeter.)
\item E-c (Given a tweet, classify it as 'neutral or no emotion' or as one, or more, of eleven given emotions that best represent the mental state of the tweeter.)
\end{enumerate}
 
For all of these subtasks, three language subsubtasks are present, for english, spanish and arabic.\\ 
\subsection{Models}
To best solve these subtasks, different models are tried out, compared and analysed. Furthermore, 2 different approaches are tried out to solve the tasks. One approach is the feature based approach which builds on the bag-of-words model (En reference måske?), which takes into account the individual words but not the context in which the words are placed in the sentence. The other approach is the deep learning(igen reference måske?) way, in which an implementation of a recurrent neural network (RNN), called a gated recurrent unit, is used.

\subsection{Data} \label{sec:introdata}
To get an understanding of the tasks to be solved, one needs to understand the way the data has been annotated. In table \ref{tab:tweets} are some exemplary tweets in the four regression emotions and their corresponding classfication labels.\\
\begin{table}[h]
\scalebox{0.8}{\begin{tabular}{c|ccc}
Regression emotion & Tweet & Regression value & Classification labels \\ 
\hline
Anger & \parbox[t]{6cm}{So my Indian Uber driver just called someone the N word. If I wasn't in a moving vehicle I'd have jumped out \#disgusted} & 0.896 & 1 0 1 0 0 0 0 0 0 0 0 \\ 
\hline
Fear & \parbox[t]{6cm}{@Wunderlist Help are you guys still down? When can we go up?} & 0.312 & 0 1 0 0 0 0 0 0 0 0 0 \\ 
\hline
Joy & \parbox[t]{6cm}{Whatever you decide to do make sure it makes you \#happy.} & 0.583 & 0 0 0 0 1 1 1 0 0 0 0 \\ 
\hline
Sadness & \parbox[t]{6cm}{Honestly don't know why I'm so unhappy most of the time. I just want it all to stop :(   \#itnevergoes} & 0.949 & 0 0 0 1 0 0 0 1 1 0 0 \\ 
\end{tabular}}
\caption{Exemplary tweets with regression and clasification labels}
\label{tab:tweets}
\end{table}
The regressional value is how much of a particular regression emotion that is felt and the classification labels are mapped as such: anger, anticipation, disgust, fear, joy, love, optimism, pessimism, sadness, surprise and trust. Since the data for the two subtasks have been annotated independently, the anger flag could be off in an tweet with a low regression value for anger. \\
Furthermore, since this PAPER was written before the release date of the test data sets, all data used in this PAPER is either train data or development data.